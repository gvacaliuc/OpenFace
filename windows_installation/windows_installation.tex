\documentclass{tufte-handout}

%\geometry{showframe}% for debugging purposes -- displays the margins

\usepackage{amsmath}

% Set up the images/graphics package
\usepackage{graphicx}
\setkeys{Gin}{width=\linewidth,totalheight=\textheight,keepaspectratio}
\graphicspath{{graphics/}}

\title{Installing OpenFace from sources in Windows}
\author{Gabriel Vacaliuc}
%\date{24 January 2009}  % if the \date{} command is left out, the current date will be used

% The following package makes prettier tables.  We're all about the bling!
\usepackage{booktabs}

% The units package provides nice, non-stacked fractions and better spacing
% for units.
\usepackage{units}

% The fancyvrb package lets us customize the formatting of verbatim
% environments.  We use a slightly smaller font.
\usepackage{fancyvrb}
\fvset{fontsize=\normalsize}

% Small sections of multiple columns
\usepackage{multicol}

% Provides paragraphs of dummy text
\usepackage{lipsum}

% These commands are used to pretty-print LaTeX commands
\newcommand{\doccmd}[1]{\texttt{\textbackslash#1}}% command name -- adds backslash automatically
\newcommand{\docopt}[1]{\ensuremath{\langle}\textrm{\textit{#1}}\ensuremath{\rangle}}% optional command argument
\newcommand{\docarg}[1]{\textrm{\textit{#1}}}% (required) command argument
\newenvironment{docspec}{\begin{quote}\noindent}{\end{quote}}% command specification environment
\newcommand{\docenv}[1]{\textsf{#1}}% environment name
\newcommand{\docpkg}[1]{\texttt{#1}}% package name
\newcommand{\doccls}[1]{\texttt{#1}}% document class name
\newcommand{\docclsopt}[1]{\texttt{#1}}% document class option name

\usepackage{listings}
\usepackage{color}

\definecolor{dkgreen}{rgb}{0,0.6,0}
\definecolor{gray}{rgb}{0.5,0.5,0.5}
\definecolor{mauve}{rgb}{0.58,0,0.82}

\lstset{frame=tb,
  language=Bash,
  aboveskip=3mm,
  belowskip=3mm,
  showstringspaces=false,
  columns=flexible,
  basicstyle={\small\ttfamily},
  numbers=none,
  numberstyle=\tiny\color{gray},
  keywordstyle=\color{blue},
  commentstyle=\color{dkgreen},
  stringstyle=\color{mauve},
  breaklines=true,
  breakatwhitespace=true,
  tabsize=3
}

\begin{document}

\maketitle% this prints the handout title, author, and date

\begin{abstract}
    \noindent This document details the installtion of OpenFace\footnote{
        https://github.com/TadasBaltrusaitis/OpenFace} in Windows 10 from source files, 
    without relying on the Visual Studio IDE.
\end{abstract}

\section{Environment Setup}

\subsection{CMake}

CMake is a powerful build tool that will set up the correct makefiles for a slew of 
setups/IDEs/compilers.  Likely the only change you'll have to make for everything in this guide 
to work with your environment is the \Verb|-G| flag to CMake.

\newthought{This document} will assume the use of CMake 3.8 (the latest as
of compilation)\footnote{ However, any CMake $> 3.0$ or so should work fine.}  Navigate to the CMake 
Download page\footnote{https://cmake.org/download/} and download the windows installer (*.msi). \\

\newthought{During the installation}, make sure to elect to add the execuatable path to the System PATH
for (at least) the your profile. \\

\subsection{Git}

It's likely you already have Git installed, but in the event you don't, navigate to Git client
download page\footnote{https://git-scm.com/downloads} and download the 64-bit windows installer. \\

\newthought{During the installation}, make sure to elect to add the execuatable path to the System PATH
for (at least) the your profile. \\

\subsection{Microsoft Visual C++ Build Tools}

\newthought{For our C/C++ compiler}, we'll install the MSVC 2017 Build
Tools\footnote{https://www.visualstudio.com/downloads/\#build-tools-for-visual-studio-2017}.
Note that if you already have Visual Studio 2017 installed, these will likely have already
been installed.  When installing, be sure to check the boxes to install the C++ compilers as well 
as C++/CLI Support.\footnote{Note that you don't have install the full VS IDE, we just need the
build tools which are found on the same website.} \\ 

\newthought{After the installation}, you should be able to navigate to the start menu and open both:
\begin{itemize}
    
    \item Developer Command Prompt for VS 2017
    \item x64 Native Tools Command Prompt for VS 2017

\end{itemize}

\newthought{For convenience}, keep a single ``x64 Native Tools Command Prompt for VS 2017`` open
for the duration of the installation.  Depending on your setup, you may have to run it as 
administrator.  This will allow us to set some environment variables to reduce typing.

\section{Dependencies}

\subsection{zlib}

The first dependency to install is zlib\footnote{http://zlib.net}.  The sources can be downloaded
from the URL to the right, making sure to download the *.zip. \\

    \newthought{After downloading}, extract the files to the directory \Verb|C:\local\|.\footnote{
    You may have to create this directory.}  This should result in a directory such as
    \Verb|C:\local\zlib-1.x.x|.

    \newthought{To build}, navigate to your zlib directory and execute the following 
    commands\footnote{Note that this guide uses \Verb|nmake| as its build tool, however cmake provides
    generators for several IDEs, see output of \Verb|cmake -h| for details.}:

    \begin{lstlisting}
        mkdir build
        cd build 
        cmake .. -G "NMake Makefiles"
        nmake
    \end{lstlisting}

    If everything worked correctly, you should have files \Verb|zlibd.lib|, \Verb|zlibd.dll| in
    your build directory.  For convenience later, set some environment variables.  Assuming that
    your zlib directory is \\
    \Verb|C:\local\zlib-1.2.11|, set:

    \begin{lstlisting}
        set ZLIB_INCLUDE_DIR=C:\local\zlib-1.2.11
        set ZLIB_LIBRARY=%ZLIB_INCLUDE_DIR%\build\zlibd.lib
    \end{lstlisting}

\subsection{libpng}

    Our second dependency depends directly on zlib.  libpng can be downloaded from 
    SourceForge\footnote{https://sourceforge.net/projects \\
    /libpng/files/libpng16/1.6.29/}, making 
    sure to download the *.zip file.

    \newthought{After downloading}, extract the files to \Verb|C:\local\|.

    \newthought{To build}, navigate to the libpng directory:
    \Verb|C:\local\lpng1629\|, and execute:

    \begin{lstlisting}
        mkdir build
        cd build
        cmake .. -DZLIB_LIBRARY=%ZLIB_LIBRARY% ^ 
                 -DZLIB_INCLUDE_DIR=%ZLIB_INCLUDE_DIR% ^
                 -G "NMake Makefiles"
        mklink %ZLIB_INCLUDE_DIR%\zconf.h %ZLIB_INCLUDE_DIR%\zconf.h.in
        nmake
    \end{lstlisting}

    \newthought{If successful}, you should the file \Verb|libpng16d.lib| in your build directory.
    For convenience later, set the following environment variables:

    \begin{lstlisting}
        set PNG_INCLUDE_DIR=C:\local\lpng1629
        set PNG_LIBRARY=%PNG_INCLUDE_DIR%\build\libpng16d.lib
    \end{lstlisting}

\subsection{libjpeg}

    The final dependency we'll build, libjpeg,  is stored on 
    github\footnote{https://github.com/stohrendorf/libjpeg-cmake.git}.

    \newthought{To download and build}, execute the following in the ongoing Command Prompt 
    from \Verb|C:\local\|:

    \begin{lstlisting}
        git clone https://github.com/stohrendorf/libjpeg-cmake.git jpeg-9a
        mkdir jpeg-9a\build
        cd jpeg-9a\build
        cmake .. -G "NMake Makefiles"
        nmake
    \end{lstlisting}

    \newthought{If successful}, you should have the file \Verb|libjpeg.lib| in your build
    directory.  For convenience later, set the following environment variables:

    \begin{lstlisting}
        set JPEG_INCLUDE_DIR=C:\local\jpeg-9a
        set JPEG_LIBRARY=%JPEG_INCLUDE_DIR%\build\libjpeg.lib
    \end{lstlisting}

\subsection{OpenCV}

    \newthought{Download} the OpenCV Windows installer from their SourceForge 
    page\footnote{https://sourceforge.net/projects/\\ opencvlibrary/files/opencv-win/3.2.0/}.  Upon
    download, run the exe and install the files to \Verb|C:\local\|.  For convenience later,
    set the following environment variable in your Command Prompt:

    \begin{lstlisting}
        set OPENCV_DIR=C:\local\opencv\build\x64\vc14\lib
    \end{lstlisting}

\subsection{Boost}

    \newthought{Download} the Boost Windows Setup from their SourceForge 
    page\footnote{https://sourceforge.net/projects/boost/files/boost-binaries/1.59.0/}.  Opt for
    the 64-bit version compiled with the latest version of MSVC (14.0).  The filename you download
    should be:\\ \Verb|boost_1_59_0-msvc-14.0-64.exe|.\\  Upon download, run the executable, and 
    have it install the files to \Verb|C:\local|.  For convenience later,
    set the following environment variables in your Command Prompt:

    \begin{lstlisting}
        set BOOST_ROOT=C:\local\boost_1_59_0
        set BOOST_LIBRARY_DIR=%BOOST_ROOT%\lib64-msvc-14.0
    \end{lstlisting}

\subsection{TBB}

    \newthought{Download} the prebuilt Windows libraries from Intel's TBB release 
    page\footnote{https://github.com/01org/tbb/releases}.  Be sure to download the file ending
    in `*win.zip'.

    \newthought{Extract} the contents of the downloaded Zip File to \Verb|C:\local|.  For
    convenience later, assuming that your extracted TBB directory is: \\
    \Verb|C:\local\tbb2017_20170412oss| \\
    set the following environment variables:

    \begin{lstlisting}
        set TBB_ROOT_DIR=C:\local\tbb2017_20170412oss
        set TBB_LIBRARY=%TBB_ROOT_DIR%\lib\intel64\vc14\tbb.lib
        set TBB_INCLUDE_DIR=%TBB_ROOT_DIR%\include
    \end{lstlisting}
        
\section{OpenFace Installation}

    \subsection{Header Fixing}
        
        CMake modifies the filename of a library header for libpng, so we'll
        create a symbolic link from the old filename to the true file.
        Assuming we have our environment variables set up, simply execute the
        commands:

        \begin{lstlisting}
            mklink %PNG_INCLUDE_DIR%\pnglibconf.h %PNG_INCLUDE_DIR%\scripts\pnglibconf.h.prebuilt
        \end{lstlisting}

    \subsection{Installation Proper}

    With all the dependencies in place, we can compile the OpenFace project.  Working
    either in the original repository\footnote{https://github.com/TadasBaltrusaitis/OpenFace} or 
    the forked repository\footnote{https://github.com/gvacaliuc/OpenFace} I'm providing,
    execute the following commands:

    \begin{lstlisting}
        mkdir build
        cd build
        # if using gvacaliuc/OpenFace
        ../build.bat
        # else using TadasB/OpenFace
        cmake .. ^
                -DOpenCV_DIR=%OPENCV_DIR% ^
                -DTBB_ROOT_DIR=%TBB_ROOT_DIR% ^
                -DTBB_LIBRARY=%TBB_LIBRARY% ^
                -DTBB_INCLUDE_DIR=%TBB_INCLUDE_DIR% ^
                -DZLIB_LIBRARY=%ZLIB_LIBRARY% ^
                -DZLIB_INCLUDE_DIR=%ZLIB_INCLUDE_DIR% ^
                -DPNG_PNG_INCLUDE_DIR=%PNG_INCLUDE_DIR% ^
                -DPNG_LIBRARY=%PNG_LIBRARY% ^
                -DJPEG_INCLUDE_DIR=%JPEG_INCLUDE_DIR% ^
                -DJPEG_LIBRARY=%JPEG_LIBRARY% ^
                -DCMAKE_BUILD_TYPE=Release ^
                -G "NMake Makefiles"
        nmake
    \end{lstlisting}

    \newthought{Note} that the above commands assume that you've preserved the same Command
    Prompt that we defined the environment variables in.  Note also that you must perform this
    compilation in an ``x64 Native Tools Command Prompt for VS 2017'', not a regular Developer
    Command Prompt.

\section{Running the Executables}

    The executables that this guide builds has a few DLL dependencies that Windows needs to
    be aware of.  In the Command Prompt you just built OpenFace in, update the \Verb|%PATH%|
    environment variable to include the OpenCV DLL, Boost DLL's, and the TBB DLL:

    \begin{lstlisting}
        set TMP_OPENCV=%OPENCV_DIR%\..
        set PATH=%PATH%;%TBB_ROOT_DIR%\bin\intel64\vc14
        set PATH=%PATH%;%BOOST_LIBRARY_DIR%
        set PATH=%PATH%;%TMP_OPENCV%\bin;
    \end{lstlisting}
   
    \break

    \newthought{After updating} the path, you should be able to run the executables from the 
    \Verb|build\bin\Release| subdirectory like:

    \begin{lstlisting}
        ...\build\bin\Release> ..\FaceLandMarkImg.exe -root . -gaze -f input.png -oi output.png
    \end{lstlisting}

\end{document}
